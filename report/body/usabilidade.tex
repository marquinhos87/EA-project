\chapter{Usabilidade}
\label{chap:usabilidade}

\hspace{5mm} Nesta secção discute-se a importância da usabilidade no desenvolvimento deste projecto, os princípios que se podem encontrar, e em que medida os mesmos trazem melhorias para a interacção com o sistema.

\section{Conhecer os utilizadores}

\hspace{5mm} Um do grandes sábios de sistemas interactivos, \href{https://www.goodreads.com/quotes/8740235-we-must-design-for-the-way-people-behave-not-for}{Donald A. Normal} disse a seguinte afirmação, \textit{"We must design for the way people behave, not for how we would wish them to behave.”}, isto significa que, para desenvolver um bom sistema interactivo deve-se seguir a ideologia de que \textbf{o programa é feito para o utilizador}. Desta forma, deve-se desenvolver os programa a pensar nas necessidades do utilizador e como o mesmo pretende \textbf{usar} (usabilidade) o sistema de forma eficiente, eficaz e satisfatória, para atingir os seus objectivos. De forma muito resumida, deve-se desenvolver/desenhar focado na usabilidade. Por este motivo, neste projecto aplicou-se a mesma ideologia, ou seja, necessita-se de conhecer os utilizadores do programa.

\hspace{5mm} O conhecimento/análise dos utilizadores contém diversas técnicas conhecidas tais como: entrevista, observação, personas, etc. No entanto, face aos recursos que a equipa tem, a forma mais eficaz de conhecer o utilizador foi através de pesquisa e estudo do domínio do problema (exercício físico/ ginásio). Na verdade, começou-se por perceber a linguagem utilizada no meio, isto é, termos como: planos de treino, workouts, series, tempos entre series, personal trainer, entre outros. Apesar de poder parecer pouco importante, isto permitiu desenvolver uma interface com uma linguagem adequada aos utilizadores. Do mesmo modo, a utilização de interfaces simplistas, como se verá mais adiante também foi pensada no utilizador, pois pessoas focadas no exercício tendem a ser simplistas. Outra razão deve-se ao facto de a tendência actual de desenvolvimento web seguir a simplicidade das interfaces (seguir as \textbf{tendências}/\textbf{padrão} torna-se muito importante para se ter uma interface \textbf{consistente} com a actualidade).


\section{Princípios de usabilidade}

\hspace{5mm} Poder-se-á dizer que o desenvolvimento de interfaces torna-se uma processo consistente, desta forma, foram definidos/partilhados, princípios de usabilidade. Pode-se dizer que os princípios de usabilidade são regras/normas que guiam os desenvolvedores/designers a desenharem boas interfaces. Desta forma, a equipa desenhou a interface do projecto focada na usabilidade e nos princípios.

\hspace{5mm} De forma sucinta, apresentar-se-á de seguida alguns dos princípios de usabilidade mais importantes presentes neste projecto, visto que os na secção \ref{chap:mockups} serão abordados com mais detalhe para cada \textbf{view}. 

\hspace{5mm} Um dos princípios presentes no projecto, e dos mais importantes no desenho de interfaces, como já foi dito anteriormente consiste na \textbf{Consistência}. A consistência a nível de linguagem, componentes de interface ao longo das várias views, em relação á actualidade onde o desenho segue a tendência/consistência actual (simplicidade) entre outros casos. 

\hspace{5mm} A \textbf{Familiarização} um pouco semelhante ao anterior, consiste na utilização de padrões, por exemplo, a apresentação de uma semana do plano segue o estilo \textbf{comum} de um calendário/horário semanal, a inserção de datas faz-se com um calendário, etc.

\hspace{5mm} Outro princípio bastante comum ao longo do projecto consiste na \textbf{Predictability}, isto é, quando se sabe que algo não pode ser feito, \textbf{impede-se} de fazer, por exemplo o botão "Guardar Semana" de uma plano, está desabilitado quando não existem workouts definidos para a mesma, pois não será possível guardar, tornando-se, desta forma, a interface mais eficiente/eficaz.

\hspace{5mm} A \textbf{Synthesizability} consiste em promover ao utilizar o conhecimento do estado do sistema, ou seja, informar o sucesso ou insucesso de acções feitas pelo mesmo. Neste projecto, pode-se encontrar este principio em todas as views, com a informação de erros, conclusão ou execução de ações (sucessos), também na criação plano, quando são adicionados workouts à semana, o personal trainer consegue perceber essa adição com a view que foi feita, tal como se verá mais adiante.

\hspace{5mm} A nível de flexibilidade, está presente o principio \textbf{Observability}, mas mais concretamente a característica de \textbf{Browsability}, para a navegação entre semanas no plano, permitindo ao cliente sentir-se ter controlo do sistema.

\hspace{5mm} Mais adiante, como já foi dito anteriormente, na secção \ref{chap:mockups} (secção seguinte) , os princípios serão avaliados para cada view com mais detalhe.

