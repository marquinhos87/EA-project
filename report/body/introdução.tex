\chapter{Introdução}
\label{chap:introducao}

\hspace{5mm} No Mestrado integrado de Engenharia Informática, no perfil de Engenharia de Aplicações, nas unidades curriculares de Arquitecturas Aplicacionais e Sistemas Interactivos foi proposto o desenvolvimento de uma aplicação web seguindo todos os princípios leccionados ao longo do semestre. 

\hspace{5mm} Ao longo do relatório apresenta-se todas as etapas efectuadas para a concretização deste projecto. \hspace{5mm} O projecto o qual intitulou-se de "\textbf{Gym@Home}"\ consiste no desenvolvimento de uma aplicação web de acompanhamentos de treino online criados por \textbf{personal trainers} e realizados por \textbf{clients}.

\hspace{5mm} Inicialmente faz-se um contextualização do problema, apresentando o domínio do mesmo, bem como os seus objectivos. 

\hspace{5mm} Após o conhecimento do problema, apresenta-se a terminologia do meio e a lista de requisitos funcionais e não funcionais do projecto bem como a sua análise, sendo bastante importante visto que reflectem as funcionalidades do sistema.

\hspace{5mm} De seguida, apresenta-se a modelação desenvolvida: modelo domínio, diagrama de classe. De forma, mais detalhada e pormenorizada em relação ao desenvolvimento, apresenta-se o diagrama de classes, com a estrutura global do sistema e as respectivas decisões e explicações necessárias sobre o mesmo, bem como identificação de problemas que de alguma forma foram prevenidos. Ainda relacionado com o diagrama de classes, também se explica a obtenção de código a partir do mesmo, bem como o código da framework \textbf{Hibernate} para a persistência de dados.

\hspace{5mm} De seguida, apresenta-se a prototipagem das interfaces, bem como o resultado final. Para cada interface explica-se o que fazem, os princípios encontrados nas mesmas, bem como as heurísticas de norman por forma a avaliar o resultado.

\hspace{5mm} Após a explicação dos vários componentes na modelação, explica-se como se faz o deployment automatizado da infraestrutura/arquitectura da aplicação utilizando docker-compose.

\hspace{5mm} Por último será feita uma avaliação geral do projecto, com objectivos atingidos, pontos a melhorar e trabalho futuro.