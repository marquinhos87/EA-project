\chapter{Conclusão}
\label{chap:conclusao}

\hspace{5mm} Em suma, conclui-se que grande parte dos objectivos inicialmente propostos foram atingidos, conseguindo-se chegar a uma aplicação web que na opinião da equipa com uma boa interface focada na usabilidade, bem como no utilizador. Do mesmo modo, a arquitectura orientada a serviços foi conseguida com sucesso, permitindo ao grupo perceber as vantagens e desvantagens das mesmas.

\hspace{5mm} Inicialmente houveram algumas dificuldades no desenho do diagrama de classes, pois não estávamos a entender se era vantajoso aplicar determinados design patterns como o \textbf{Builder}, \textbf{Factory}, entre outros, que após algumas sessões de dúvidas foram se dissipando.

\hspace{5mm} Depois a equipa teve complicações a usar as ferramentas \textbf{NetBeans} e \textbf{IntelliJ} para a criação de beans. No entanto, após um pouco de pesquisa percebeu-se que existia um problema nos ficheiros de configuração.

\hspace{5mm} Após a conclusão dos serviços foi necessário fazer a infraestrutura, bem como o deployment, no entanto, foi nesse momento que o grupo percebeu que a arquitectura é bastante pesada, correndo apenas num computador de um elemento do grupo, limitando-nos para testar pois estávamos dependentes do mesmo. No entanto encontrou-se uma solução de executar os serviços em localhost, todos partilhando o mesmo servidor web, ficando muito mais leve, permitindo que já não houvesse tanta dependência.

\hspace{5mm} A nível de frameworks apenas usou-se o Hibernate, de resto foi tudo o que se aprendeu nas aulas, pois são as ferramentas que o grupo se sentiu mais à vontade, tais como JSP, Servlets, Beans, ...

\hspace{5mm} Uma das interfaces, Criar uma semana, na primeira apresentação estava demasiado complexa, onde o grupo foi avisado pela equipa docente do mesmo. Dessa forma, a equipa decidiu redesenhar o mockup com apenas uma página, conseguindo-se o resultado acima, que na opinião do grupo ficou "perfeito".

\hspace{5mm} Após a utilização do Hibernate gerado pelo Visual Paradigm, o grupo ficou um pouco reticente à utilização do mesmo neste formato, pois "tira-nos" um bocado liberdade, se o projecto começasse hoje provavelmente o grupo optaria por fazer com Hibernate em anotações, ou outra framework ou até desenvolver a comunicação com a BD.

\hspace{5mm} Como em todos os projectos existe sempre pontos a melhorar e/ou acrescentar, pelo que este projecto não foi excepção. De seguida são enumerados alguns desses pontos.

\begin{itemize}
    \item Introduzir pagamentos na aplicação, para os Clientes poderem pagar pelos seus planos e consequentemente os PTs receberem pelo seu trabalho.
    
    \item Colocar passagem de tempo nas tarefas para o Cliente não necessitar de utilizar cronómetros ou outros dispositivos para controlar a passagem do tempo.
    
    \item Adicionar um histórico de planos no Cliente para este poder visualizar todos os planos que já realizou.
    
    \item Pedir o certificado de PT aos PTs para estes se poderem registar na aplicação, uma vez que é necessário controlar se quem cria planos está devidamente certificado para o fazer.

    \item Adicionar novo tipo de utilizador (Administrador) para validar os certificados introduzidos pelos PTs.
    
    \item Acrescentar Workouts predefinidos para os PTs não terem de preencher a mesma informação várias vezes.
    
    \item Para uma melhor experiência do utilizador, acrescentar paginação nas notificações, nos pedidos realizados (Cliente), nos pedidos recebidos (PT) e nos clientes actuais (PT).
    
    \item De forma a melhorar ainda mais a experiência dos utilizadores, fazer o redimensionamento das interfaces para dispositivos mais pequenos. 
    
\end{itemize}

\hspace{5mm} Assim, conclui-se, que este projecto foi importante para se perceber a importância quer de desenhar um boa arquitectura e as consequências do mesmo, quer da importância do desenvolvimento da interface gráfica focada nos objectivos do utilizador para maximizar a usabilidade. 