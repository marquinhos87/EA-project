\chapter{Levantamento e Análise de Requisitos}
\label{chap:requisitos}

\hspace{5mm} Na presente secção serão ilustrados os requisitos levantados e analisados para o problema em questão. Existindo dois tipos de utilizadores os requisitos estão divididos em três partes, a parte comum a ambos, a parte do Cliente e a parte do PT.

\section{Requisitos Comuns}
\label{sec:requisitoscomuns}

\begin{itemize}
    \item \textbf{Login}: os utilizadores necessitam de se autenticar para usufruir do sistema.
    
    \item \textbf{Logout}: os utilizadores podem fazer logout para terminar a sessão na plataforma.
    
    \item \textbf{Registar}: os utilizadores necessitam de se registar no sistema.
    
    \item \textbf{Visualizar Perfil}: os utilizadores poderão visualizar o seu perfil e o perfil de outros utilizadores.
    
    \item \textbf{Editar Perfil}: os utilizadores poderão editar os seus dados para os manter actualizados.
    
    \item \textbf{Ver Notificações}: os utilizadores podem visualizar  notificações para saber acções que outros utilizadores tomaram no sistema relacionadas com ele.
    
    \item \textbf{Visualizar Semana}: os utilizadores podem visualizar a semana, tendo a perspectiva de todos os workouts para essa semana e também a sua condição física através dos dados biométricos do Cliente.
    
    \item \textbf{Visualizar Workout}: os utilizadores podem visualizar o workout, vendo assim as tarefas desse workout.
    
\end{itemize}

\section{Requisitos Cliente}
\label{sec:requisitoscliente}

\begin{itemize}
    \item \textbf{Realizar Workout}: o Cliente realiza workouts.
    
    \item \textbf{Procurar Personal Trainer}: o Cliente procura por PTs podendo especificar alguns filtros para uma selecção mais refinada de um PT.
    
    \item \textbf{Fazer um Pedido}: o Cliente preenche um formulário com o tipo de plano que pretende realizar.
    
    \item \textbf{Avaliar Personal Trainer}: o Cliente no final do Plano pode avaliar o PT.
    
    \item \textbf{Visualizar Pedidos Feitos}: o Cliente visualiza todos os pedidos que realizou a PTs.
    
\end{itemize}

\section{Requisitos Personal Trainer}
\label{sec:requisitospt}

\begin{itemize}
    \item \textbf{Criar Semana}: o PT cria uma semana para ao plano do cliente.
    
    \item \textbf{Responder a Pedido}: o PT aceita ou rejeita pedidos efectuados pelos clientes.
    
    \item \textbf{Visualizar Clientes}: o PT vê os seus clientes onde poderá obter informações sobre os seus dados pessoais e acrescentar semanas ao seu plano.
    
    \item \textbf{Visualizar Pedidos Recebidos}: o PT visualiza os pedidos pendentes enviados pelos Clientes.
    
\end{itemize}

\hspace{5mm} O diagrama de Use Cases não está presente, pois este é directamente análogo aos requisitos levantados. O diagrama de Use Cases estaria dividido em dois subsistemas, que seria o do Cliente e o do PT, sendo que no do PT estariam todos os Use Cases respectivos aos requisitos comuns e do PT, no subsistema do Cliente estariam os requisitos comuns e do Cliente.